% !TEX program = xelatex
\documentclass{article}
\usepackage{amsfonts}
\usepackage{xepersian}
\settextfont[Scale=1]{XB Niloofar}
\author{github.com/parasilius/moral-letters-to-lucilius}
\begin{document}
\section*{۳. در باب دوستی حقیقی و غیرحقیقی}
    \subsection*{۱}
    چندی پیش برای من به واسطه‌ی یکی از دوستانت(به قول خودت) نامه‌ای فرستادی.
    اما در همان جمله‌ی بعدی نامه به من هشدار دادی که درباره‌ی تو با وی چیزی در میان نگذارم، دقیقاً همانند خودت؛
    به بیانی دیگر در همان نامه هم دوستی او را تأیید و هم رد کرده‌ای.
    \subsection*{۲}
    حال اگر این کلمه‌ی ما\footnote{کلمه‌ای که نزد رواقیون از اهمیت زیادی برخوردار است؛ منظور همان کلمه‌ی دوستی است.} را در معنای عام آن به کار بردی، و او را ((دوست)) نامیدی همان‌گونه که ما از همه‌ی داوطلبان انتخابات به عنوان ((افراد محترم)) یاد می‌کنیم، و یا اگر نام شخصی که به طور معمول او را می‌بینیم از یاد ببریم وی را ((آقا/خانم محترم)) حطاب می‌‌کنیم، ایرادی ندارد.
    اما اگر هر کسی را که مثل خودت به او اعتماد نداری دوست خودت بدانی، سخت در اشتباهی و معنای دوستی حقیقی را آن طور که باید درک نکرده‌ای.
    در واقع من تو را به در میان گذاشتن هر چیزی با یک دوست تشویق می‌کنم، اما اول درمورد خود دوست تصمیم بگیر.
    وقتی پایه‌ی دوستی شکل می‌گیرد، باید اعتماد کنی؛ قبل از آن باید داوری کنی.
    آن‌هایی که با نادیده گرفتن اصول تئوفراستوس\footnote{Theophrastus} بعد از آنکه شخصی را به عنوان دوست خود برگزیده‌اند او را داوری می‌کنند به جای آنکه بعد از داوری او را دوست خود بگیرند، آخر را بر اول ترجیح داده و وظایفشان را به جای هم انجام داده‌اند.
    برای مدت طولانی درمورد انتخاب یک نفر به عنوان دوست خود بیندیش؛ اما وقتی دوستی‌اش را قبول کردی، با تمام وجود و از صمیم قلب به وی خوشامد بگو.
    بی‌واهمه با او سخن بگو همان طور که با خودت سخن می‌گویی.
    \subsection*{۳}
    هرچند باید طوری زندگی کنی که هیچ چیزی را که حتی به دشمن خودت نمی‌توانی بسپاری به خودت هم برای نگهداری آن اعتماد نکنی، از آنجا که برخی چیزها را عرفاً باید به صورت راز نگه داشت، حداقل باید همه‌ی نگرانی‌هایت را با یک دوست در میان بگذاری.
    او را باوفا و معتمد بدار، و با این کار باوفا خواهد شد.
    مثلاً بوده‌اند افرادی که از ترس فریب خوردن به باقی آدم‌ها فریب دادن را آموزش داده‌اند؛
    با شک و تردیدهایشان به دوست خود حق داده‌اند که خطا کند.
    چرا باید کلمه‌ای را در حضور دوستم پنهان کنم؟
    چرا نباید در حضور دوستم خودم را تنها بپندارم؟
    \subsection*{۴}
    دسته‌ای از افراد هستند که با هرکسی که ملاقات می‌کنند، آنچه را که فقط باید برای دوستان افشا شود بیان می‌کنند، و آنچه که آن‌ها را می‌آزارد برای هر شنونده‌ی تصادفی می‌گویند.
    از طرفی عده‌ای هستند که می‌ترسند رازهایشان را به نزدیک‌ترین دوست‌هایشان بگویند؛ و حتی در صورت امکان، به خودشان هم اعتماد نکرده و رازهایشان را در اعماق دلشان دفن می‌کردند.
    اما ما نباید هیچ کدام از این دو کار را انجام دهیم.
    به همان اندازه اشتباه است به همه اعتماد کنی که به هیچ کس اعتماد نداشته باشی.
    با این حال باید بگویم که اولین اشتباه\footnote{اعتماد به همه} بچگانه‌تر و دومین اشتباه\footnote{اعتماد به هیچ کس} امن‌تر است.
    \subsection*{۵}
    به طریق مشابه باید هم آن آدم‌هایی را که آرامش ندارند و هم آن‌هایی که همواره در آرامش هستند را نکوهش کنی.
    چراکه عشق به کار زیاد و بیش از حد فعالیت نیست، بلکه نشان از یک ذهن شکار شده است.
    آرامش حقیقی آن نیست که هر کاری را فقط مایه‌ی رنجش دانسته و نکوهش کرد؛ بلکه این تنبلی است.\footnote{سنکا درمورد برقرار کردن تعادل بین کار کردن و آرامش صحبت می‌کند؛ اما مسئله بین کار کردن و آرامش است و نه کار کردن و کار نکردن. همچنین بازدید از صفحه‌ی زیر هم می‌تواند مفید واقع شود:\newline \begin{LTR}https://dailystoic.com/make-as-in-a-few-things/\end{LTR}}
    \subsection*{۶}
    پس این گفته را که در مطالعاتم از پمپونیوس\footnote{Pomponius} آورده‌ام در نظر داشته باش: ((برخی آدم‌ها به گوشه‌های تاریکی فرو می‌روند، آنچنان که حتی روز هم سیاهی می‌بینند.))
    نه، بلکه انسان‌ها باید این تمایلات را با هم ترکیب کنند؛ او که آرامش دارد باید عمل کرده و او که عمل می‌کند باید آرامش را دریابد.
    اگر این مسئله را با طبیعت\footnote{Nature} در میان بگذاری،‌ به تو می‌گوید که هم روز و هم شب را پدید آورده است.
    خدانگهدار.
\end{document}