% !TEX program = xelatex
\documentclass{article}
\usepackage{amsfonts}
\usepackage{xepersian}
\settextfont[Scale=1]{XB Niloofar}
\author{github.com/parasilius/moral-letters-to-lucilius}
\begin{document}
\section*{۴۳. در باب نسبی بودن شهرت}
    \subsection*{۱}
    از خودت می‌پرسی چگونه اخبار به من رسیدند و چه کسی مرا از چیزی که در ذهن خودت می‌پروراندی باخبر کرد، حال آنکه درمورد آن به هیچ کس چیزی نگفته بودی؟
    در واقع همان اطلاعات غالب افراد بود؛ غیبت.
    می‌گویی: ((چطور ممکن است، آیا من چنان شخصیت مهمی هستم که پشت سرم صحبت می‌کنند؟))
    اما دلیلی ندارد که خودت را نسبت به این بخش از جهان\footnote{روم} بسنجی.
    همان قسمتی را در نظر بگیر که هم‌اکنون در آن زندگی می‌کنی.
    \subsection*{۲}
    هر نقطه‌ای که از نقاط همسایه فراتر می‌رود در همان جایی که هست دارای عظمت است.
    چراکه عظمت مطلق نیست؛
    بلکه با قیاس افزوده شده و یا کاهش می‌یابد.
    آن کشتی که در رودخانه بزرگ می‌نماید، در اقیانوس بسیار کوچکتر به نظر می‌رسد.
    سکانی که برای یک کشتی بزرگ و جاگیر است، برای دیگری کوچک است.
    \subsection*{۳}
    پس تو در استان خودت\footnote{لوسیلیوس در آن زمان والی سلطنتی سیسیل بود.} دارای اهمیت زیادی هستی، هرچند خود را کوچک بپنداری.
    دیگران دنبال آن هستند که بفهمند تو چه می‌کنی، چگونه شام صرف می‌کنی و چگونه می‌خوابی، و جواب‌هایشان را هم پیدا می‌کنند.
    پس دلیل مضاعفی است برای تو که با احتیاط بیشتری زندگی کنی.
    با این حال تا زمانی که بتوانی آشکارا درمیان مردم زندگی کنی، آن موقع که دیوارها تو را مخفی نکرده بلکه از تو محافظت می‌کنند، خود را خوشبخت حقیقی نپندار؛
    هرچند ما تمایل داریم که باور کنیم این دیوارها اطرافمان را پوشانده نه برای آنکه با امنیت بیشتر زندگی کنیم، بلکه برای آنکه مخفیانه‌تر گناه کنیم.
    \subsection*{۴}
    بگذار حقیقتی را بگویم که با آن می‌توانی ارزش شخصیت آدمی را بسنجی:
    به ندرت کسی را می‌یابی که می‌تواند درحالی که در خانه‌اش رو به عموم باز است زندگی کند.
    این باطن ماست و نه غرورمان که دربانان را در جلوی درهایمان گماشته است؛
    آنچنان زندگی می‌کنیم که اگر ناگهان درهایمان به عموم باز شود انگار در میانه‌ی عملی گیرمان انداخته‌اند.
    با این حال چه سودی برایمان دارد که خود را پنهان کرده و از چشم‌ها و گوش‌های دیگران مخفی داریم؟
    \subsection*{۵}
    باطن پاک از عموم استقبال می‌کند، حال آنکه نفس آلوده حتی در تنهایی آشفته و پریشان است.
    اگر کردارت شریف باشد بگذار همه در جریان باشند؛
    اما اگر اعمالت پست و فرومایه‌اند، چه تفاوتی دارد که کسی نداند، حال آنکه خودت می‌دانی.
    چقدر خوار هستی اگر از چنین شاهدی بیزار باشی!\footnote{و خود را از بقیه کمتر بشماری.}
    بدرود.
\end{document}